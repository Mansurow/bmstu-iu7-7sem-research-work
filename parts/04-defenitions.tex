%\specsection{ОПРЕДЕЛЕНИЯ}
\chapter*{ОПРЕДЕЛЕНИЯ}
\addcontentsline{toc}{chapter}{ОПРЕДЕЛЕНИЯ}

\begin{description}[leftmargin=0pt]
	\item Пакет --- это блок данных, содержащий информацию, необходимую для доставки~\cite{tcp_ip_reilly}.
	\item Задержка --- время, которое требуется для передачи данных отправки данных по Сети~\cite{tcp_ip_reilly}. 
	\item Джиттер --- изменчивость или нестабильность во времени задержки при передаче данных~\cite{tcp_ip_reilly}.
	\item Пропускная способность --- максимально возможное количество данных, которые могут быть переданы через канал за некоторый промежуток времени~\cite{tcp_ip_reilly}.
	\item Мультиплексирование  --- образование из нескольких отдельных потоков общего агрегированного потока, который может быть передан по одному физическому каналу связи~\cite{tcp_ip_lora}.
	\item Компьютерная сеть --- набор связанных между собой автономных компьютеров. Два компьютера называются связанными между собой, если они могут обмениваться информацией~\cite{network_tanenbaum}.
	\item Канал передачи данных --- канал электросвязи для передачи сигналов данных~\cite{network_tanenbaum}.
	\item Трансляция  --- непрерывная передача аудио или видеоданнах с источника на платформу для дальнейшего распространнения~\cite{basic_stream_protocls}.
	\item Ретрансляция --- повторная передача видеоданных без изменения его содержимого~\cite{basic_stream_protocls}.
	\item Видеопоток --- последовательность видеоданных, которая передается по компьютерным сетям или другим каналам связи~\cite{basic_stream_protocls}.
	\item Протокол --- набор соглашений логического уровня.
	\item Потоковая передача данных (передача потоков) --- способ передачи, при котором транспортировка и воспроизведение мультимедийных данных на удаленном компьютере осуществляются в режиме реального времени~\cite{basic_stream_protocls}. 

	\item Соединение --- канал транспортного уровня, установленный между двумя программами с целью обмена данными~\cite{rfc_rtsp}.
	\item Сеанс --- механизм передачи непрерывного медиапотока или запуск потока с помощью данных управления~\cite{rfc_rtsp}.
	\item Битрейт --- количество бит, используемых для передачи или обработки данных в единицу времени~\cite{hls_apple}.
\end{description}

