\chapter*{ОБОЗНАЧЕНИЯ И СОКРАЩЕНИЯ}
\addcontentsline{toc}{chapter}{ОБОЗНАЧЕНИЯ И СОКРАЩЕНИЯ}

В текущей расчетно-пояснительной записке применяется следующие сокращения и обозначения.

\begin{description}[leftmargin=0pt]
	\item \noindent ISO --- International Standards Organization, международной организацией по стандартизации.
	\item \noindent OSI --- Open Systems Interconnect, взаимодействия открытых систем.
	\item \noindent TCP --- Transmission Control Protocol,  протокол управления передачей.
	\item \noindent UDP --- User Datagram Protocol, протокол пользовательских дейтаграмм.
	\item \noindent PAR --- Positive Acknowledgment with Retransmission.
	\item \noindent RTP --- Real--Time Transport Protocol, транспортный протокол реального времени.
	\item \noindent RTCP --- Real--Time Transport Control Protocol, протокол управления передачей в реальном времени.
	\item \noindent RTSP --- Real--Time Streaming Protocol, потоковый протокол реального времени.
	\item \noindent RTMP --- Real--Time Messaging Protocol, протокол обмена сообщениями в реальном времени.
	\item \noindent HTTP --- HyperText Transfer Protocol, протокол передачи гипертекста.
	\item \noindent HLS --- HTTP Live Streaming, прямая передача потоковых данных по протоколу HTTP.
	\item \noindent MPEG-DASH --- Dynamic Adaptive Streaming over HTTP, технология адаптивной потоковой передачи данных.
	\item \noindent MPTCP --- Multipath TCP, многопутевой TCP. 
	\item \noindent ABR --- Adaptive Bit--Rate Streaming, трансляция с адаптивынм битрейтом.
	\item \noindent QoS --- Quality of Service, качество обслуживания.
\end{description}

