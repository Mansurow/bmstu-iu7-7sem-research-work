\chapter*{ВВЕДЕНИЕ}
\addcontentsline{toc}{chapter}{ВВЕДЕНИЕ}

Одной из тенденций развитий современного общества является развитие интeрнет-технологии, которые проникли в каждый аспект жизни человека.
Ежедневно растет количество устройств и платформ с поддержкой видеоконтента, критически важных предприятий и государственных служб зависят от корректной работы Сети~\cite{mediascope_2022}.

Поэтому повышаются требования к качеству и надежности передачи видеоинформации всем пользователям.
Провайдеры сетей стремятся обеспечить стабильность  передачи информации, улучшая существующую инфраструктуру, но существуют факторы которые могут оказать негативное влияние на процесс передачи видеопотоков~\cite{research_videocontent_2021, research_videocontent_2023}:
\begin{itemize}
	\item нагрузка на сеть в текущий момент времени;
	\item устаревшая или неэффективная инфраструктура;
	\item нестабильное соединение может привести к потере пакетов данных;
	\item физические помехи и условия окружающей среды;
	\item низкая пропускная способность.
\end{itemize}

Таким образом, чтобы доставить видеоинформацию произвольному количеству пользователей возникает необходимость организации ретрансляции видеопотоков от устройства--захвата для обеспечения стабильной и надежной передачи видеоинформации.
Организация ретрансляции видеопотоков становится ключевым шагом для решения этих проблем. Путем передачи трансляции видео через устройства--захвата на сервере, задача которого повторно передавать или хранить видеоматериал для повторной передачи пользователям.

Целью работы является проведение анализа алгоритмов организации ретрансляции видеопотоков.

\clearpage

Чтобы достигнуть поставленной цели, необходимо решить следующие задачи:
\begin{itemize}
	\item провести анализ предметной области ретрансляции видеопотоков;
	\item провести обзор существующих алгоритмов организации ретрансляции видеопотоков;
	\item сформулировать критерии сравнения алгоритмов организации ретрансляции видеопотоков;
	\item классифицировать существующие алгоритмы организации ретрансляции видеопотоков.  
\end{itemize}
